%
% Introductrory Software Development - Laboratory 21
%

% Latex 2e

\pdfcompresslevel=9
\pdfoutput=1

\pdfcompresslevel=9
\pdfoutput=1

\documentclass[12pt,oneside]{cttutorial}
\usepackage[pdftex]{graphicx}
\usepackage{etoolbox}
\usepackage{hyperref}
\usepackage[procnames]{listings}
\usepackage{color}
\usepackage{url}
\DeclareGraphicsExtensions{.pdf,.png,.mps}

\hypersetup{%
  pdfauthor={Professor Colin Turner},%
  pdftitle={Introductory Software Development Laboratory 17},%
  pdfkeywords={Python, Software},%
  pdfproducer={LaTeX},%
}

\newbool{questions_only}
\ifdefined\questionsonly
  \setbool{questions_only}{true}
\else
  \setbool{questions_only}{false}
\fi

\begin{document}

\modulenumber{EEE203}
\moduletitle{Introductory Software Development}
\modulelecturer{Professor Turner}
\tutorialword{Laboratory}
\tutorialnumber{17}
\tutorialextra{}


\definecolor{keywords}{RGB}{255,0,90}
\definecolor{comments}{RGB}{0,0,113}
\definecolor{red}{RGB}{160,0,0}
\definecolor{green}{RGB}{0,150,0}
 
\lstset{language=Python, 
        basicstyle=\ttfamily\small, 
        keywordstyle=\color{keywords},
        commentstyle=\color{comments},
        stringstyle=\color{red},
        showstringspaces=false,
        identifierstyle=\color{green},
        procnamekeys={def,class},
        backgroundcolor=\color[rgb]{0.95, 0.95, 0.95},
	lineskip=-1pt
}

\newcommand{\xkcd}[2]{
	\begin{center}
	\includegraphics[scale = 0.5]{../../Figures/png/#1}
	\newline
	\url{http://xkcd.com/#2}
	\end{center}
	\bigskip
}

\newcommand{\alert}[1]
{\marginpar
  {\makebox[0 pt][l]
    {\includegraphics[scale=0.1]{../../Figures/png/warning.png}
  }
  \parbox{2 cm}{{\sffamily \bfseries \tiny #1}}}}


\renewcommand{\baselinestretch}{1.5}
\textwidth=15cm
% Over-full v-boxes on even pages are due to the \v{c} in author's name
\vfuzz2pt % Don't report over-full v-boxes if over-edge is small

\newcommand{\I}{j}

\begin{center}
\begin{bfseries}
Introductory Software Development\\Laboratory 17 - Numerical Exercises
\end{bfseries}
\end{center}

\section{Aims \& Objectives}

The aims of this lab are:

\begin{itemize}
\item to consolidate algorithm construction;
\item to consolidate working with control structures.
\end{itemize}

\section{Divisors}

\alert{Start a NEW file}
Proposed name: lab17\_1.py
Write a python script with a function called \lstinline!divisors()! which will return, for a positive integer, all the integer divisors (those numbers that divide into the original number with no remainder). The returned value should be in list form.

The function should check the input for validity (that it is indeed a positive integer, and return an empty list if not).

\section{Prime Numbers}

Add a function called \lstinline!prime_numbers()! that takes a positive integer as an argument and returns a list of all the prime numbers up to and including that number.

Remember, a prime number has exactly two divisors, itself and one.

\section{Prime Factors}

Add a function called \lstinline!prime_factors()! that works like \lstinline!divisors()! except that only prime number factors are returned.

\section{Pythagorean Triples}

A \emph{Pythagorean Triple} is a set of three positive integers $a$, $b$ and $c$ such that
\[
a^2 + b^2 = c^2
\]

There are infinitely many such triples.

\subsection{Checking A Triple}

Write a function called \lstinline!check_pythagorean_triple()! that takes three positive integer variables and checks if they are a Pythagorean Triple.

\subsection{Generating Triples}

Can you create a function \lstinline!generate_pythagorean_triples()! that will print out a certain number of triples for an input $n$?

Can you adjust it to print the triples in some sort of order?

\section{Fibonacci Sequences}

The standard Fibonacci sequence is

\[
1,1,2,3,5,8,13,21,34, \ldots
\]

Write a Python function called \lstinline!Fibonacci()! that will output all Fibonacci numbers up to and including a positive integer used as input.


\section{Extra Credit}

\begin{itemize}
\item Ensure all functions are properly documented, commented and appropriately robust.

\item Consider improving the speed of your Prime Number finder by implementing the Sieve of Eratosthenes. \url{https://en.wikipedia.org/wiki/Sieve_of_Eratosthenes}

\item Review previous labs you may not have completed as preparation for the final test.
\end{itemize}


\end{document}

\newpage
\section{Sample solutions}

\end{document}



