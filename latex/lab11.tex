%
% Introductrory Software Development - Laboratory 11
%

% Latex 2e

\pdfcompresslevel=9
\pdfoutput=1

\pdfcompresslevel=9
\pdfoutput=1

\documentclass[12pt,oneside]{cttutorial}
\usepackage[pdftex]{graphicx}
\usepackage{etoolbox}
\usepackage{hyperref}
\usepackage[procnames]{listings}
\usepackage{color}
\usepackage{url}
\DeclareGraphicsExtensions{.pdf,.png,.mps}

\hypersetup{%
  pdfauthor={Professor Colin Turner},%
  pdftitle={Introductory Software Development Laboratory 11},%
  pdfkeywords={Python, Software},%
  pdfproducer={LaTeX},%
}

\newbool{questions_only}
\ifdefined\questionsonly
  \setbool{questions_only}{true}
\else
  \setbool{questions_only}{false}
\fi

\begin{document}

\modulenumber{EEE203}
\moduletitle{Introductory Software Development}
\modulelecturer{Professor Turner}
\tutorialword{Laboratory}
\tutorialnumber{9}
\tutorialextra{}


\definecolor{keywords}{RGB}{255,0,90}
\definecolor{comments}{RGB}{0,0,113}
\definecolor{red}{RGB}{160,0,0}
\definecolor{green}{RGB}{0,150,0}
 
\lstset{language=Python, 
        basicstyle=\ttfamily\small, 
        keywordstyle=\color{keywords},
        commentstyle=\color{comments},
        stringstyle=\color{red},
        showstringspaces=false,
        identifierstyle=\color{green},
        procnamekeys={def,class},
        backgroundcolor=\color[rgb]{0.95, 0.95, 0.95},
	lineskip=-1pt
}

\newcommand{\xkcd}[2]{
	\begin{center}
	\includegraphics[scale = 0.5]{../../Figures/png/#1}
	\newline
	\url{http://xkcd.com/#2}
	\end{center}
	\bigskip
}

\newcommand{\alert}[1]
{\marginpar
  {\makebox[0 pt][l]
    {\includegraphics[scale=0.1]{../../Figures/png/warning.png}
  }
  \parbox{2 cm}{{\sffamily \bfseries \tiny #1}}}}


\renewcommand{\baselinestretch}{1.5}
\textwidth=15cm
% Over-full v-boxes on even pages are due to the \v{c} in author's name
\vfuzz2pt % Don't report over-full v-boxes if over-edge is small

\newcommand{\I}{j}

\begin{center}
\begin{bfseries}
Introductory Software Development\\Laboratory 11 - Random Relief
\end{bfseries}
\end{center}

\section{Aims \& Objectives}

The aims of this lab are:

\begin{itemize}
\item to understand how to generate pseudo random numbers;
\item to build very simply games;
\item to have some fun after the class tests.
\end{itemize}


\section{Random Numbers}

In general computers do not produce truly random numbers, but instead use pseudo random numbers. Python uses the \lstinline!random! module to do this.

\subsection{Guess a number}

\alert{Start a NEW file}
Proposed name: lab11\_1.py
\begin{lstlisting}
# Experiments with random numbers
# Put your name here

import random

def guess_my_number(lowest=1, highest=100):
    # Generate the number
    number = random.randrange(lowest, highest+1)

    guessed = False
    while not guessed:
        guess = int(input("What number am I thinking of? "))

        if guess > number:
            print("Lower")
        elif guess < number:
            print("Higher")
        else:
            print("Correct!")
            guessed = True

\end{lstlisting}

You should add a \lstinline!main()! function that is called in the usual way and which calls this function. Can you work out the fastest strategy to work out the correct answer?

\subsection{Rock Paper Scissors}

\alert{Start a NEW file}
Proposed name: lab11\_2.py
\begin{lstlisting}
# Experiments with random numbers
# Put your name here

import random

def rock_paper_scissors():
    choices = ['rock','paper','scissors']

    print("Pick one of the following")
    index = 1
    for choice in choices:
        print("{}. {}".format(index, choice))
        index += 1

    number = int(input("Pick 1, 2 or 3: "))
    # Python uses 0 for the first item in a list, subtract 1
    user_choice = choices[number-1]
    print("You picked {}".format(user_choice))

    # Ok, now the computer choice, we can use a function
    # random.choice() to pick one item at random from a list
    computer_choice = random.choice(choices)
    print("I picked {}".format(computer_choice))
         
\end{lstlisting}

Once again, add a function \lstinline!main()!. Can you also complete the programme to work out who won depending on the possible outcomes?

For example:

\begin{lstlisting}
    if user_choice == 'rock' and computer_choice == 'rock':
        print("A draw!")
\end{lstlisting}
and so on.

Can you then add two new choices, lizard and Spock? \url{https://en.wikipedia.org/wiki/Rock-paper-scissors#Additional_weapons}.



\section{Extra Credit}

Can you write a script to produce lucky dip numbers for the lottery?

\end{document}

\newpage
\section{Sample solutions}

\end{document}



