%
% Introductrory Software Development - Laboratory 18
%

% Latex 2e

\pdfcompresslevel=9
\pdfoutput=1

\pdfcompresslevel=9
\pdfoutput=1

\documentclass[12pt,oneside]{cttutorial}
\usepackage[pdftex]{graphicx}
\usepackage{etoolbox}
\usepackage{hyperref}
\usepackage[procnames]{listings}
\usepackage{color}
\usepackage{url}
\DeclareGraphicsExtensions{.pdf,.png,.mps}

\hypersetup{%
  pdfauthor={Professor Colin Turner},%
  pdftitle={Introductory Software Development Laboratory 18},%
  pdfkeywords={Python, Software},%
  pdfproducer={LaTeX},%
}

\newbool{questions_only}
\ifdefined\questionsonly
  \setbool{questions_only}{true}
\else
  \setbool{questions_only}{false}
\fi

\begin{document}

\modulenumber{EEE203}
\moduletitle{Introductory Software Development}
\modulelecturer{Professor Turner}
\tutorialword{Laboratory}
\tutorialnumber{18}
\tutorialextra{}


\definecolor{keywords}{RGB}{255,0,90}
\definecolor{comments}{RGB}{0,0,113}
\definecolor{red}{RGB}{160,0,0}
\definecolor{green}{RGB}{0,150,0}
 
\lstset{language=Python, 
        basicstyle=\ttfamily\small, 
        keywordstyle=\color{keywords},
        commentstyle=\color{comments},
        stringstyle=\color{red},
        showstringspaces=false,
        identifierstyle=\color{green},
        procnamekeys={def,class},
        backgroundcolor=\color[rgb]{0.95, 0.95, 0.95},
	lineskip=-1pt
}

\newcommand{\xkcd}[2]{
	\begin{center}
	\includegraphics[scale = 0.5]{../../Figures/png/#1}
	\newline
	\url{http://xkcd.com/#2}
	\end{center}
	\bigskip
}

\newcommand{\alert}[1]
{\marginpar
  {\makebox[0 pt][l]
    {\includegraphics[scale=0.1]{../../Figures/png/warning.png}
  }
  \parbox{2 cm}{{\sffamily \bfseries \tiny #1}}}}


\renewcommand{\baselinestretch}{1.5}
\textwidth=15cm
% Over-full v-boxes on even pages are due to the \v{c} in author's name
\vfuzz2pt % Don't report over-full v-boxes if over-edge is small

\newcommand{\I}{j}

\begin{center}
\begin{bfseries}
Introductory Software Development\\Laboratory 18 - Bluetooth
\end{bfseries}
\end{center}

\section{Aims \& Objectives}

The aims of this lab are:

\begin{itemize}
\item to demonstrate how to find and request service lists from Bluetooth devices;
\item to practice building simple interfaces to code.
\end{itemize}

\section{Introduction}

You will need a computer with a working Bluetooth stack in order to make this code work. You may need to get a working dongle to do this.

You may also need to install the pybluez Python module. You can usually do this by typing

\begin{lstlisting}
pip install pybluez
\end{lstlisting}

at a command line. For Linux and OS X, you will want to open a terminal to do this. You may need to preface your command with \lstinline!sudo! depending on your environment. On Windows, try going to the start menu, select run and then type \lstinline!cmd! to get your terminal.

\alert{Note} You might need to use \lstinline!pip3! rather than \lstinline!pip! especially if your machine has both Python 2.x and Python 3.x installed at the same time.

\section{Finding Devices}

\alert{Start a NEW file}
Proposed name: lab18\_1.py

\begin{lstlisting}
# Experiments with Python and Bluetooth

import bluetooth

# Obtain the Bluetooth addresses of nearby devices
devices = bluetooth.discover_devices()

for device in devices:
  print(device)
\end{lstlisting}

Enable Bluetooth on your phone and put your phone into discoverable mode.\footnote{If you don't have a phone, see if you can join forces with someone else.} Try running the script.

If it doesn't find the bluetooth module and the install went OK, you might need to restart IDLE first.

Assuming that went OK, now try:

\begin{lstlisting}
# Experiments with Python and Bluetooth

import bluetooth

# Obtain the Bluetooth addresses of nearby devices
devices = bluetooth.discover_devices()

for device in devices:
  print(device, bluetooth.lookup_name(device))
\end{lstlisting}

Can you find your own phone? If so, note it's bluetooth address. This is unique to that device.

\section{Service Discovery}

Each Bluetooth device can advertise a number of services. We can ask it to provide that list. Add
some code to your program like this:

\begin{lstlisting}
services = bluetooth.find_service(address=device)
for service in services:
    print("Name        {}".format(service["name"]))
    print("Description {}".format(service["description"]))
    print("Protocol    {}".format(service["protocol"]))
    print("Port        {}".format(service["port"]))
    print()
\end{lstlisting}

This code only prints out the Bluetooth services for the last item that was scanned. Why is this?


\section{Improving the Listing}

Create two functions, one of which scans the available nearby Bluetooth devices and allows the user to select one of them.

The second function should take the device of a Bluetooth device and check for its available services.

Comment and document-comment your listing appropriate.

Can you move these two parts into separate functions and write a program that allows the user to select which of the discovered devices should be checked.





%\end{document}

\newpage
\section{Sample solutions}

\begin{lstlisting}
a# Work with Bluetooth

# We will need the pyBluez library installed
import bluetooth


def SelectBluetoothDevice():
    """Scan for the Bluetooth devices in the area and allow
    one to be picked. The function returns a Bluetooth
    address, or None if no device is picked."""

    print("Please wait, scanning for devices...")
    # Scan for the devices around
    devices = bluetooth.discover_devices()

    # Show them in a list, with an index
    index = 0
    print("List of devices with addresses and names.")
    for device in devices:
        print(index, device, bluetooth.lookup_name(device))
        index+=1

    index = int(input("Pick a device to check: "))
    if (index < 0 or index >= len(devices)):
        print("Invalid choice.")
        return None
    else:
        # It's a valid choice, return the BT address
        return devices[index]


def ShowBluetoothServices(device):
    """Takes a Bluetooth address and shows the
    services available"""

    print("You choose", device, bluetooth.lookup_name(device))

    print("Looking for services...")
    services = bluetooth.find_service(address=device)
    for service in services:
        print("Name        {}".format(service["name"]))
        print("Description {}".format(service["description"]))
        print("Protocol    {}".format(service["protocol"]))
        print("Port        {}".format(service["port"]))
        print()


def main():
    """Function to select a Bluetooth device and
    print its services"""
    
    # Keep going until interrupted by Ctrl-C
    while True:
        # Select a device 
        device = SelectBluetoothDevice()
        if device == None:
            # No device was available or no valid one chosen
            print("No device chosen.")
        else:
            # Show the services for the device
            ShowBluetoothServices(device)
        
if __name__ == '__main__':
    main()

\end{lstlisting}

\section{Extra Credit}

This code has no resilience checking for the Bluetooth code calling, add this for extra credit.

\end{document}



