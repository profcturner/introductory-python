%
% Introductrory Software Development - Laboratory 20
%

% Latex 2e

\pdfcompresslevel=9
\pdfoutput=1

\pdfcompresslevel=9
\pdfoutput=1

\documentclass[12pt,oneside]{cttutorial}
\usepackage[pdftex]{graphicx}
\usepackage{etoolbox}
\usepackage{hyperref}
\usepackage[procnames]{listings}
\usepackage{color}
\usepackage{url}
\DeclareGraphicsExtensions{.pdf,.png,.mps}

\hypersetup{%
  pdfauthor={Professor Colin Turner},%
  pdftitle={Introductory Software Development Laboratory 20},%
  pdfkeywords={Python, Software},%
  pdfproducer={LaTeX},%
}

\newbool{questions_only}
\ifdefined\questionsonly
  \setbool{questions_only}{true}
\else
  \setbool{questions_only}{false}
\fi

\begin{document}

\modulenumber{EEE203}
\moduletitle{Introductory Software Development}
\modulelecturer{Professor Turner}
\tutorialword{Laboratory}
\tutorialnumber{20}
\tutorialextra{}


\definecolor{keywords}{RGB}{255,0,90}
\definecolor{comments}{RGB}{0,0,113}
\definecolor{red}{RGB}{160,0,0}
\definecolor{green}{RGB}{0,150,0}
 
\lstset{language=Python, 
        basicstyle=\ttfamily\small, 
        keywordstyle=\color{keywords},
        commentstyle=\color{comments},
        stringstyle=\color{red},
        showstringspaces=false,
        identifierstyle=\color{green},
        procnamekeys={def,class},
        backgroundcolor=\color[rgb]{0.95, 0.95, 0.95},
	lineskip=-1pt
}

\newcommand{\xkcd}[2]{
	\begin{center}
	\includegraphics[scale = 0.5]{../../Figures/png/#1}
	\newline
	\url{http://xkcd.com/#2}
	\end{center}
	\bigskip
}

\newcommand{\alert}[1]
{\marginpar
  {\makebox[0 pt][l]
    {\includegraphics[scale=0.1]{../../Figures/png/warning.png}
  }
  \parbox{2 cm}{{\sffamily \bfseries \tiny #1}}}}


\renewcommand{\baselinestretch}{1.5}
\textwidth=15cm
% Over-full v-boxes on even pages are due to the \v{c} in author's name
\vfuzz2pt % Don't report over-full v-boxes if over-edge is small

\newcommand{\I}{j}

\begin{center}
\begin{bfseries}
Introductory Software Development\\Laboratory 20 - Correct Faults
\end{bfseries}
\end{center}

\section{Aims \& Objectives}

The aims of this lab are:

\begin{itemize}
\item to practice identifying faults in code.
\end{itemize}

\section{Example}

What is wrong with the following example.

\begin{lstlisting}
print "hello world"
\end{lstlisting}

Is there a reason why code like this would be common to see?

\section{Example}

Can you find the fault in this code to print a dividing line?

\begin{lstlisting}
number = 20

divider = "*" * Number
print(divider)
\end{lstlisting}



\section{Example}

How many faults are in this piece of code. Can you fix them?

\begin{lstlisting}
number1 = input("Enter a first number"))
number2 = input("Enter a second number"))

print("The total is", number + number2)
\end{lstlisting}

\section{Example}

Can you find the fault in the piece of code?

\begin{lstlisting}
1number = int(input("Enter a number"))
2number = int(input("Enter another number"))

3number = 1number + 2number
\end{lstlisting}

\section{Example}

How many faults are in this piece of code? Can you fix them?

\begin{lstlisting}
for n in range(0,11)
  total = total + range
  
print total
\end{lstlisting}

\section{Example}

How many faults are these in this piece of code? Can you fix them?

\begin{lstlisting}
x = input("x + 2 = 4. What is x?")

if x = 2:
  print("Yes.")
else
  print("No.")
\end{lstlisting}

\section{Example}

What is the main problem with this piece of code?

\begin{lstlisting}
for=input("What is the reason for your message")

if for == "notice":
  print("Thank you")
\end{lstlisting}

\section{Example}

There are a number of mistakes in this code. Can you find and fix them?

\begin{lstlisting}
# Calculate 1 x 2 x 3 .. x 10
product = 0
for n in range(0,9):
  product = product * n
  
print("The answer is", n)
\end{lstlisting}



\end{document}

\newpage
\section{Sample solutions}

\end{document}



