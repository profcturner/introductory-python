%
% Introductrory Software Development - Laboratory 12
%

% Latex 2e

\pdfcompresslevel=9
\pdfoutput=1

\pdfcompresslevel=9
\pdfoutput=1

\documentclass[12pt,oneside]{cttutorial}
\usepackage[pdftex]{graphicx}
\usepackage{etoolbox}
\usepackage{hyperref}
\usepackage[procnames]{listings}
\usepackage{color}
\usepackage{url}
\DeclareGraphicsExtensions{.pdf,.png,.mps}

\hypersetup{%
  pdfauthor={Professor Colin Turner},%
  pdftitle={Introductory Software Development Laboratory 12},%
  pdfkeywords={Python, Software},%
  pdfproducer={LaTeX},%
}

\newbool{questions_only}
\ifdefined\questionsonly
  \setbool{questions_only}{true}
\else
  \setbool{questions_only}{false}
\fi

\begin{document}

\modulenumber{EEE203}
\moduletitle{Introductory Software Development}
\modulelecturer{Professor Turner}
\tutorialword{Laboratory}
\tutorialnumber{12}
\tutorialextra{}


\definecolor{keywords}{RGB}{255,0,90}
\definecolor{comments}{RGB}{0,0,113}
\definecolor{red}{RGB}{160,0,0}
\definecolor{green}{RGB}{0,150,0}
 
\lstset{language=Python, 
        basicstyle=\ttfamily\small, 
        keywordstyle=\color{keywords},
        commentstyle=\color{comments},
        stringstyle=\color{red},
        showstringspaces=false,
        identifierstyle=\color{green},
        procnamekeys={def,class},
        backgroundcolor=\color[rgb]{0.95, 0.95, 0.95},
	lineskip=-1pt
}

\newcommand{\xkcd}[2]{
	\begin{center}
	\includegraphics[scale = 0.5]{../../Figures/png/#1}
	\newline
	\url{http://xkcd.com/#2}
	\end{center}
	\bigskip
}

\newcommand{\alert}[1]
{\marginpar
  {\makebox[0 pt][l]
    {\includegraphics[scale=0.1]{../../Figures/png/warning.png}
  }
  \parbox{2 cm}{{\sffamily \bfseries \tiny #1}}}}


\renewcommand{\baselinestretch}{1.5}
\textwidth=15cm
% Over-full v-boxes on even pages are due to the \v{c} in author's name
\vfuzz2pt % Don't report over-full v-boxes if over-edge is small

\newcommand{\I}{j}

\begin{center}
\begin{bfseries}
Introductory Software Development\\Laboratory 12 - Classes to end Classes
\end{bfseries}
\end{center}

\section{Aims \& Objectives}

The aims of this lab are:

\begin{itemize}
\item to understand simple concepts of class and objects;
\item to build and use very simple classes.
\end{itemize}


\section{Classes}

Classes allow variables and code to be wrapped up together. We have been using classes for quite a long time now, but it's time to start learning to make our own.

\alert{Start a NEW file}
Proposed name: lab12\_1.py
\begin{lstlisting}
# Experiments with classes
# Put your name here

class Person(object):
    ''' Describes a person

    name    the name of the person
    age     the age of the person
    '''
    def __init__(self):
        self.name = ''
        self.age = 0


def main():
    joe = Person()

   print(joe.name)
   print(joe.age)
   
    joe.name = 'Joe Bloggs'
    joe.age = 25

    print(joe.name)
    print(joe.age)
    print(joe)

if __name__ == '__main__':
    main()
    
\end{lstlisting}

Please run this script. You will see that we create a \emph{class} called \lstinline!Person! and an \emph{object} called \lstinline!joe! in the function \lstinline!main()!.

The object behaves like any other variable, but we can use the \lstinline!.! operator to access the component parts of it, either to set the values, or to the read them.

You will notice that the \lstinline!__init__! function within the class is called automatically when the object is created, and sets up the values inside. This is a \emph{constructor}.

You will also notice the last \lstinline!print! command provides something a bit messy. At the IDLE prompt after running the script, type

\begin{lstlisting}
help(Person)
\end{lstlisting}

\subsection{\_\_str\_\_}

Next, try to add another function inside the class. Be careful to copy the indentation of the existing \lstinline!__init__! function. As the underscores suggest, this function name is a bit special.

\begin{lstlisting}
    def __str__(self):
        return self.name + ' age: ' + str(self.age)
\end{lstlisting}

Run the script again. You will see the last \lstinline!print()! function does something more sensible now.

This is the Pythonic way of teaching Python how to print an object, or more accurately, what to do if someone tries to convert an object into a string, which is what happens when \lstinline!print()! is called.

\subsection{Adding member variables}

The variables inside classes are special and are called \emph{members} or \emph{attributes}.

Can you add another member variable called \lstinline!email! to store email addresses into the class, and amend the documentation, and \lstinline!main()! appropriately?

\subsection{Adding member functions}

In Python, member functions are usually called \emph{methods}.

Add another method to the \lstinline!Person! class.

\begin{lstlisting}
    def input(self):
        self.name = input('Enter a name :')
        self.email = input('Enter an email address :')
        self.age = int(input('Enter the age :'))
\end{lstlisting}

To call this, change the lines that set the variables inside \lstinline!joe! in \lstinline!main()! to

\begin{lstlisting}
joe.input()
\end{lstlisting}

\subsection{Docstrings}
Complete useful docstrings for the methods in your class.

\section{Writing another class}

Write another class called \lstinline!Point! which has three attributes, \lstinline!x!, \lstinline!y! and \lstinline!z! that represents a point in three dimensions.

It should include \lstinline!__init__()!, \lstinline!__str__()! and \lstinline!input()! methods that do sensible things.

Test these in your \lstinline!main()! function.

\section{Extra Credit}

Review the previous labs and see if you can see examples where classes would help.

Have a good Christmas break. We will delve deeper into this next year.
\end{document}

\newpage
\section{Sample solutions}

\end{document}



